\documentclass[conference]{IEEEtran}
\IEEEoverridecommandlockouts
% The preceding line is only needed to identify funding in the first footnote. If that is unneeded, please comment it
% out.
\usepackage[numbers]{natbib}
\bibliographystyle{plainnat}
\usepackage{hyperref}
\usepackage{amsmath,amssymb,amsfonts}
\usepackage{algorithmic}
\usepackage{graphicx}
\graphicspath{ {./assets/} }
\usepackage{textcomp}
\usepackage{xcolor}
\usepackage{lipsum}
\def\BibTeX{{\rm B\kern-.05em{\sc i\kern-.025em b}\kern-.08em
T\kern-.1667em\lower.7ex\hbox{E}\kern-.125emX}}
\begin{document}


    \title{Floating aquatic vegetation recognition using self-supervised methods with remote sensing data}

    \author{\IEEEauthorblockN{Esteve Soria Fabián}
    %\IEEEauthorblockA{\textit{dept. name of organization (of Aff.)} \\
    \textit{Universidad Internacional Menéndez Pelayo}\\
    esofabian@gmail.com}
    %\and
    %\IEEEauthorblockN{2\textsuperscript{nd} Given Name Surname}
    %\IEEEauthorblockA{\textit{dept. name of organization (of Aff.)} \\
    %\textit{name of organization (of Aff.)}\\
    %City, Country \\
    %email address or ORCID}

    \maketitle

    \begin{abstract}
        The Copernicus programme~\cite{whatiscopernicus} is an Earth observation component of the European Union Space
        Programme.
        One of the missions inside the programme is sentinel-2~\cite{sentinel-2} with 2 satellites orbiting around the
        earth.
        The two satellites, Sentinel-2A and Sentinel-2B, can provide images in 13 different bands with a resolution up
        to 10m per pixel.
        Bands range from infrared through the visible spectrum to short wave infrared.
        The programme with its policy of free access allows scientists and researchers to obtain data for several
        research fields as cropland, glacier or lake monitoring.

        In this work we focus our attention at water quality monitoring.
        During the study of several lakes in the Spanish territory, patches of an invasive specie were found floating in the water.
        Finding out when, where and why these plants appear is of great interest for researchers.
        Its detection is a manual process through the use of common remote-sensing indices as NDVI~\cite{NDVIsource}
        and water segmentation.
        The problem of this approach is that its not very robust for a global scale search.

        This work proposes an automatic search system of these form of life with a little amount of labelled training data.
        This is thanks to the use of self-learning techniques applied to remote sensing.
    \end{abstract}
    \newline
    Code: \href{https://github.com/sorny92/satellite-surface-algae-detection}{github.com/sorny92/satellite-surface-algae-detection}.
    \newline

    \begin{IEEEkeywords}
        remote-sensing, computer vision, deep-learning, image retrieval, Sentinel-2, self-supervision
    \end{IEEEkeywords}


    \section{Introduction}
    Thanks to the globally connected world we live in its possible to communicate with people from distant countries or
    enjoy food from produce that is not native to our area.
    Yet, this global economy also has some disadvantages.
    Due to the global transport network and the big distances we can travel, humans have become an important vector of movement of species
    between different ecosystems~\cite{invasive_species}.
    Non-native species can become threads to biological ecosystems and disrupt all sort of environments,
    from lakes or forests to whole countries~\cite{bhlitem21490}.

    For many years researchers have been tracking and studying different invasive vegetation~\cite{huang2009applications, aguir2013, donyana1, donyana2} species.
    In this project we seek to automatically detect the location of aquatic vegetation floating in the water.
    The motivation is to provide a tool that could help the detection of this vegetation.
    But also provide a tool to be able to detect any other type of species or even locate other kind of images thanks to the
    high level information that provides the model generated in this work.

    In this work we use self-supervised techniques to train a model to generate embeddings from a patch of a tile.
    Embeddings are a numerical representation of some information, in our case, an image.
    Embeddings allow to compress the information contained in an image to a much manageable and representative list of values.

    We use BarlowTwins~\cite{barlowtwins} framework to train an embeddings generating architecture for remote-sensing.
    We chose to use BarlowTwins for its simple architecture which does not require a momentum encoder (like MoCo~\cite{he2020momentum, grill2020bootstrap}).
    Also, it does not need negative sampling as SimCLR~\cite{chen2020simple} which becomes helpful with satellite imaging as it allows to use
    all the world indiscriminately without having to create contrastive triplets.
    The simple architecture allows for a training that can happen in low budget systems and a minimal labelled amount of data.

    For the use of the system, fine-tuning a classification head becomes trivial with only hundreds of images with positive and negative
    samples of the target classes.
    Although, because the model can generate embeddings from any sentinel-2 image it cn also be used as a tool for image retrieval
    where the researcher can provide example images of what is looking for and the system will return the closest images that
    have been indexed.

    For this project, images have been sourced from the pair of satellites Sentinel-2~\cite{sentinel-2}.
    These satellites haven been operating since 2015 providing remote sensing capabilities in the range of latitudes between 56ºS and 84ºN\@.
    This satellite can pass above the same point every 5 days allowing researchers and companies to track specific locations
    to study how they change on time.
    This satellite has 13 different bands or wavelengths with a resolution up to 10m per pixel.
    Bands range from infrared through the visible spectrum to short wave infrared.
    The indexation of several bands together end up being really useful for some task as, for example, monitoring vegetation~\cite{TUCKER1979127}.
    Another use is the tracking of water levels of different lakes or the effect of flooding and natural catastrophes.

    Remote sensing has its own set of tools, approaches and problems.
    Some research is approached as a traditional computer vision problem were the researcher will
    generate their own heuristics to be able to track the metric they are interested into.
    An example of this is the NDVI~\cite{NDVIsource} metric which uses the bands from near infrared and red to
    measure the leaf water content and chlorophyll content~\cite{TUCKER1979127}.

    One of the problems of the heuristic approach is it own simplicity, it is excellent at measuring
    vegetation, but it also comes with its own problems.
    NDVI for example, is quite sensitive to the amount of atmosphere there is between the capture system and the target and
    the behaviour compared between different capture system is not constant~\cite{Huang2021}.

    Machine learning techniques can be useful to deal with the problems of heuristic based methods as they can fit the best model
    based on the data provided of the problem.

    Thanks to the increasing amount of satellites and aerial programs, there are many different providers with remote sensing capabilities.
    Later on in section~\ref{sec:dataset} to dive deeper in some of the data available today but nevertheless here some details of the Sentinel-2 platform
    which is used in this project:
    \begin{itemize}
        \item 4 bands at 10 meters resolution per pixel.
        \item 6 bands at 20 meters resolution per pixel.
        \item 3 bands at 60 meters resolution per pixel.
        \item Range of latitudes between 56ºS and 84ºN.
        \item Visit the same location with the same angle every 10 days or 5 days using two satellites.
        \item Tiles of 290km in size.
        \item 10980x10980 pixels of resolution.
    \end{itemize}
    The amount of data available is quite big, the problem is gathering insights of it as there's not as much labelled data
    available.
    For this project a small dataset had to be gathered, so we could benchmark how effective this framework is to detect the
    floating vegetation.

    In this project we study an improved pipeline for ease of detection of this vegetation in lakes but also provide a tool
    to find other regions of interest and similar to them.
    Currently, there are some lakes that are being tracked in a manual way where the researchers would apply several filters to detect them.
    This research makes use of data from the Cedillo dam in Spain spanning from the town of Cedillo to Alcántara.
    In figure~\ref{fig:satellite-image-airbus} an example of the vegetation mat can be seen.
    Other researchers have work in similar projects in Doñana National park (Spain)~\cite{donyana1, donyana2} or California state (USA)~\cite{rs14133013}.
    These approaches can't scale to more lakes without more people looking for them as the tools currently used are targeted
    for specific use cases and heuristics need to be developed for each individual case.
    So doing a search at a global scale can not even be considered.

    \begin{figure}[h]
        \centering
        \includegraphics[width=6cm]{figure_algae_airbus}
        \caption{Aerial image of an aquatic vegetation floating mat in the Cedillo lake (Spanish-Portuguese border). It can easily be identified as it's distinctive
        intense green colour in comparison of the water. Images from Airbus, CNES/Airbus, IGP/ DGRF 2023}
        \label{fig:satellite-image-airbus}
    \end{figure}


    \section{Related work}
    A big concern is the damage floating vegetation create for the biodiversity of aquatic areas and how this mats can cover the sunlight for species that live underwater.
    There are authors that have work in the detection and tracking of these mats~\cite{donyana1, donyana2,rs14133013, srilanka_veg, 10.3389/fmars.2022.1004012} in different areas of the world.

    There are several approaches that can be followed

    Authors like~\citet{srilanka_veg, 10.3389/fmars.2022.1004012} use pixel level classification models with manually annotated data from satellite image but also UAVs.
    Depending on the authors, there's different classes they use as target for the classification system.
    For example \citet{rs14133013} uses 8 classes: 4 different surface level species, one for submerged vegetation, one for non-photosynthetic vegetation, soil and water.
    To label the data the authors generate polygons of areas in the water based on GPS data from in-situ measurements and use GIS technologies to match them with the images from the satellite.

    \citet{10.3389/fmars.2022.1004012} also follow a similar approach taking measurements of the reflectance of the different classes in situ to then match this values to the aerial images.
    Then they use a SVM based classification technique to train a model due to the good performance they offer with low amounts of data~\cite{Cortes1995}.




    \subsection*{Remote sensing for vegetation detection}


    \subsection*{Image segmentation}
    This problem can be solved in different ways.
    The general idea is give the researchers more data of these bodies, so they can analyse where, why and when they will appear but also how do they
    behave on time.

    Image segmentation arises as an obvious approach to the problem.
    We have images of the whole earth if we can have enough images with ground truths we could train a system to segment this bodies in
    each tile so as the satellite is moving around the earth an automated system could run when these
    tiles are being generated in real-time.

    The problem of this approach is th generation of enough labelled data to become useful.
    Image segmentation has been traditionally expensive to label due to the nature of the label, as
    you need a value per pixel.
    This can be reduced with new approaches as Segment Anything~\cite{kirillov2023segment} which allows segmenting any type of image based on priors.
    These priors could be points or bounding boxes provided as input.
    This tool is becoming really useful for research and development as it can accelerate the labelling time for image segmentation tasks.
    In our case, as the dataset contains of few images with tenths of bounding boxes, SAM could be used to find the perimeter
    of the vegetation mats and be used as inputs to train a segmentation model.

    \begin{figure}[h]
        \centering
        \includegraphics[width=6cm]{segmented_tile}
        \caption{On the left a crop of a tile from sentinel-2. On the right a segmented image of the crop showing the
        location of the vegetation mat.}
        \label{fig:tile-segmented}
    \end{figure}

    \subsection{Image retreival}

    \subsection{Image classificacion}


    \section{Dataset}\label{sec:dataset}
    Other datasets
    https://torchgeo.readthedocs.io/en/stable/api/datasets.html#non-geospatial-datasets

    To work on this problem a small dataset has been gathered with 140 labelled points.
    This dataset has been captured manually by the researchers at Universidad de Valencia.
    The labelled data consists in table with a ROI defined by it's coordinates and the date
    of the image where it was located.
    With that information the image can be extracted from the sentinel-2 data

    The sentinel-2 data can be loaded with the use of EOReader~\cite{eoreader_paper}

    Eurosat dataset~\cite{helber2019eurosat} is a sentinel-2 dataset.


    \section{Approach}

    \subsection{Model pretraining}

    \subsection{Classification model}

    \subsection{Evaluation}



    Doing self-supervised learning could be useful to fight the data scarcity.
    As the dataset will be small, a self-training step will be useful to generate
    a powerful embedding space to the train a classifer:
    https://github.com/satellite-image-deep-learning/techniques#self-supervised-unsupervised--contrastive-learning
    https://ermongroup.github.io/blog/tile2vec/
    https://github.com/vladan-stojnic/CMC-RSSR

    With the embeddings of the patch maybe and MLP per pixel can classify the thingy.


    the previous idea is cool but SAM exists now so maybe it can be used to segment the algae patch
    using a point inside the ROI we have.

    the we can direct the search of more patches based on the surrounding water.
    The hypothesis is that water has X conditions that create the growth of this so if this conditions
    are meet then the should this floaty thing somewhere in this water mass.


    Comparison with other pretrained models as moco form torchgeo:
    https://torchgeo.readthedocs.io/en/stable/api/models.html#torchgeo.models.resnet50 which is trained with MOCO and https://github.com/zhu-xlab/SSL4EO-S12 dataset.


    \section{Main results}


    \section{Future work}
    Use more data as for example BigEarth dataset, or others listed to increase pretraining time thus variety of the dataset.

    Use modern architectures based on transformers as they show to be better in some cases.

    Others SSL training systems as DINOv2~\cite{DINO} show potential to stable trainings.

    Distillation to improve the execution time.

    Does the image retreival scale to other areas of the world? The system is trained with data of a limited area of the world.
    If this was tested in an area with different weather conditions or different time of the year, would it perform in a similar way?


    \section{Conclusions}

    \section*{Acknowledgment}

    \bibliography{refs}


\end{document}
