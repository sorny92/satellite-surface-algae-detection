\documentclass[conference]{IEEEtran}
\IEEEoverridecommandlockouts
% The preceding line is only needed to identify funding in the first footnote. If that is unneeded, please comment it
% out.
\usepackage[authoryear, numbers]{natbib}
\bibliographystyle{plain}
\usepackage{hyperref}
\usepackage{amsmath,amssymb,amsfonts}
\usepackage{algorithmic}
\usepackage{graphicx}
\usepackage{textcomp}
\usepackage{xcolor}
\usepackage{lipsum}
\def\BibTeX{{\rm B\kern-.05em{\sc i\kern-.025em b}\kern-.08em
T\kern-.1667em\lower.7ex\hbox{E}\kern-.125emX}}
\begin{document}


    \title{Satellite surface algae detection}

    \author{\IEEEauthorblockN{Esteve Soria Fabián}
    %\IEEEauthorblockA{\textit{dept. name of organization (of Aff.)} \\
    \textit{Universidad Internacional Menéndez Pelayo}\\
    esofabian@gmail.com}
    %\and
    %\IEEEauthorblockN{2\textsuperscript{nd} Given Name Surname}
    %\IEEEauthorblockA{\textit{dept. name of organization (of Aff.)} \\
    %\textit{name of organization (of Aff.)}\\
    %City, Country \\
    %email address or ORCID}

    \maketitle

    \begin{abstract}
        The Copernicus programme~\cite{whatiscopernicus} is an Earth observation component of the European Union Space
        Programme.
        One of the missions inside the programme is sentinel-2~\cite{sentinel-2} with 2 satellites orbiting around the
        earth.
        The two satellites, Sentinel-2A and Sentinel-2B, can provide images in 13 different bands with a resolution up
        to 10m per pixel.
        Bands range from infrared through the visible spectrum to short wave infrared.
        The programme with its policy of free access allows scientists and researchers to obtain data for several
        research fields as cropland, glacier or lake monitoring.

        In this work we focus our attention at water quality monitoring.
        During the study of several lakes in the Spanish territory, patches of an invasive specie were found floating in the water.
        Finding out when, where and why these plants appear is of great interest for researchers.
        Its detection is a manual process through the use of common remote-sensing indices as NDVI~\cite{NDVIsource}
        and water segmentation.
        The problem of this approach is that its not very robust for a global scale search.

        This work proposes an automatic search system of these form of life with a little amount of labelled training data.
        This is thanks to the use of self-learning techniques applied to remote sensing.
        Code is available at: \href{https://github.com/sorny92/satellite-surface-algae-detection}{github.com/sorny92/satellite-surface-algae-detection}.
    \end{abstract}

    \begin{IEEEkeywords}
        remote-sensing, computer vision, deep-learning, image retrieval
    \end{IEEEkeywords}


    \section{Introduction}
    During
    The goal of the project is to be able to automatically detect the location of these plants
    floating in the water.
    These bodies have been located in a few locations of Spain as X, Y and Z. The figures X and Y
    show how this bodies of plants look like from satellite.
    There is some available research where the authors map similar bodies of floating aquatic vegetation ~\cite{rs14133013}

    The source of the images is the pair of satellites Sentinel 2. This satellites haven been operationg since X year
    providing remote sensing capabilities from the range X, Y of the earth.
    This satellite can pass above the same point every X days allowing researchers and companies to
    track specific locations to study how they change on time.
    This satellite has 12 different bands or wavelengths where it can see. The bands are X, blah blah
    each one of them are useful for different fields.
    Examples of this are for example the tracking of crops as plant can be measured as X.
    Another use is the tracking of water levels of different lakes or the effect of flooding and
    natural catastrophes.

    Remote sensing has it's own set of tools, approaches and problems.
    Some research is approached as a traditional computer vision problem were the researcher will
    generated their own heuristics to be able to track the metric they are interested into.
    An example of this is the NVDI metric which is a simple calculation around some of the bands of
    sentinel which results in a relly useful tool to identify clorophil based life and how they
    propagate.
    One of the problems of the previous approach is it own simplicity, it's really good at measuring
    plants but can provide not really useful values for other cases that are not related.
    Machine learning techniques can be useful for remote sensing to avoid the problem of false
    positives in simple heuristics.

    Satellite imaging provides an enormous quantity of data.
    To be specific about the sentinel-2 platform, it generates what is called tiles, of the same coordinates.
    These images have a 1:1 aspect ratio with 10000 pixels in each dimension.
    There are 13 spectral bands: 4 bands at 10 meters, 6 bands at 20 meters and 3 bands at 60 meters of
    resolution per pixel.
    The amount of labelled data is at the scale of the whole earth, the problem is curating this
    data to something useful.

    This research is trying to improve the detection rate of this bodies of plants in lakes.
    Currently there are some lakes that are being tracked in a manual way where the researchers
    would apply several filters to detect them.
    This approach can't scale to more lakes without more operators and not even consider analysing this
    bodies at a global scale.


    \section{Related work}


    \subsection*{Image segmentation}
    This problem can be solved in different ways. The general idea is give the researchers more data
    of this bodies so they can analyse where, why and when they will appear but also how do they
    behave on time.

    Image segmentation arises as an obvious approach to the problem. We have images of the whole earth
    if we can have enough images with ground truths we could train a system to segment this bodies in
    each tile so as the satellite is moving around the earth an automated system could run when these
    tiles are being generated in real-time.

    The problem of this approach is th generation of enough labelled data to become useful.
    Image segmentation has been traditionally expensive to label due to the nature of the label, as
    you need a value per pixel.
    This can be reduced with new approaches as SAM. SAM allows segmenting anything based on a prior
    value. This prior value can be a point or a bounding box. This tool results really useful as
    the dataset we posses consists of few images with tenths of boundingboxes or the ROI.
    SAM could be used to find the edges of this bounding box so then the problem can be approached
    as a segmentation.

    \subsection{Image retreival}

    \subsection{Image segmentation}

    \subsection{Image classificacion}


    \section{Dataset}
    Other datasets
    https://torchgeo.readthedocs.io/en/stable/api/datasets.html#non-geospatial-datasets

    To work on this problem a small dataset has been gathered with 140 labelled points.
    This dataset has been captured manually by the researchers at Universidad de Valencia.
    The labelled data consists in table with a ROI defined by it's coordinates and the date
    of the image where it was located.
    With that information the image can be extracted from the sentinel-2 data

    The sentinel-2 data can be loaded with the use of EOReader~\cite{eoreader_paper}

    Eurosat dataset~\cite{helber2019eurosat} is a sentinel-2 dataset.


    \section{Approach}

    \subsection{Model pretraining}

    \subsection{Classification model}

    \subsection{Evaluation}



    Doing self-supervised learning could be useful to fight the data scarcity.
    As the dataset will be small, a self-training step will be useful to generate
    a powerful embedding space to the train a classifer:
    https://github.com/satellite-image-deep-learning/techniques#self-supervised-unsupervised--contrastive-learning
    https://ermongroup.github.io/blog/tile2vec/
    https://github.com/vladan-stojnic/CMC-RSSR

    With the embeddings of the patch maybe and MLP per pixel can classify the thingy.


    the previous idea is cool but SAM exists now so maybe it can be used to segment the algae patch
    using a point inside the ROI we have.

    the we can direct the search of more patches based on the surrounding water.
    The hypothesis is that water has X conditions that create the growth of this so if this conditions
    are meet then the should this floaty thing somewhere in this water mass.


    Comparison with other pretrained models as moco form torchgeo:
    https://torchgeo.readthedocs.io/en/stable/api/models.html#torchgeo.models.resnet50 which is trained with MOCO and https://github.com/zhu-xlab/SSL4EO-S12 dataset.


    \section{Main results}


    \section{Future work}
    Use more data as for example BigEarth dataset, or others listed to increase pretraining time thus variety of the dataset.

    Use modern architectures based on transformers as they show to be better in some cases.

    Others SSL training systems as DINOv2 show potential to stable trainings.

    Distillation to improve the execution time.

    Does the image retreival scale to other areas of the world? The system is trained with data of a limited area of the world.
    If this was tested in an area with different weather conditions or different time of the year, would it perform in a similar way?


    \section{Conclusions}

    \section*{Acknowledgment}

    \bibliography{refs}


\end{document}
