\documentclass[notitlepage]{report}
\usepackage[left=1in, right=1in, top=1in, bottom=1in]{geometry}

\usepackage{titling}
\usepackage{lipsum}

\pretitle{\begin{center}\Huge\bfseries}
\posttitle{\par\end{center}\vskip 0.5em}
\preauthor{\begin{center}\Large\ttfamily}
\postauthor{\end{center}}
\predate{\par\large\centering}
\postdate{\par}

\title{Satellite surface algae detection}
\author{Esteve Soria}
%\date{\today}
\begin{document}

    \maketitle
    \thispagestyle{empty}

    \begin{abstract}
        \lipsum[1]
    \end{abstract}


    \section*{Introduction}


    \section*{Related work}
    Interesting libraries for loading images from Sentinel-2:

    https://github.com/sertit/eoreader
    \newline
    Maybe:

    https://pro.arcgis.com/en/pro-app/latest/arcpy/get-started/what-is-arcpy-.htm
    \newline
    This looks really nice as it also allows to overlay on maps natively:

    https://geopandas.org/en/stable/getting\_started/introduction.html
    \newline
    This project is lit!

    https://github.com/satellite-image-deep-learning/techniques

    https://github.com/acgeospatial/awesome-earthobservation-code
    \newline
    Lots of datasets!

    https://github.com/Seyed-Ali-Ahmadi/Awesome\_Satellite\_Benchmark\_Datasets
    \newline
    Interesting survery here:

    https://github.com/VIROBO-15/Transformer-in-Remote-Sensing
    \newline
    To visualize custom maps:

    https://python-visualization.github.io/folium/quickstart.html

    https://github.com/giswqs/geemap
    \newline
    Too many resources for satellite imaging:

    https://github.com/sacridini/Awesome-Geospatial
    \newline
    Is this really useful?

    https://github.com/microsoft/torchgeo
    \newline
    Hmmm video results....

    https://github.com/acgeospatial/awesome-earthobservation-code#visualisation

    \subsection*{Image segmentation}


    \section*{Method}

    Doing self-supervised learning could be useful to fight the data scarcity.
    As the dataset will be small, a self-training step will be useful to generate
    a powerful embedding space to the train a classifer:
    https://github.com/satellite-image-deep-learning/techniques#self-supervised-unsupervised--contrastive-learning
    https://ermongroup.github.io/blog/tile2vec/
    https://github.com/vladan-stojnic/CMC-RSSR
    
    With the embeddings of the patch maybe and MLP per pixel can classify the thingy.
    
    
    the previous idea is cool but SAM exists now so maybe it can be used to segment the algae patch
    using a point inside the ROI we have.
    
    the we can direct the search of more patches based on the surronding water. 
    The hypothesis is that water has X conditions that create the growth of this so if this conditions
    are meet then the should this floaty thing somewhere in this water mass.


    \section*{Experiments}


    \section*{Conclusions}


\end{document}
