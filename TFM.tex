\documentclass[notitlepage]{report}
\usepackage[left=1in, right=1in, top=1in, bottom=1in]{geometry}

\usepackage{titling}
\usepackage{lipsum}

\pretitle{\begin{center}\Huge\bfseries}
\posttitle{\par\end{center}\vskip 0.5em}
\preauthor{\begin{center}\Large\ttfamily}
\postauthor{\end{center}}
\predate{\par\large\centering}
\postdate{\par}

\title{Satellite surface algae detection}
\author{Esteve Soria}
%\date{\today}
\begin{document}

    \maketitle
    \thispagestyle{empty}

    \begin{abstract}
        \lipsum[1]
    \end{abstract}


    \section*{Introduction}
    The goal of the project is to be able to automatically detect the location of these plants 
    floating in the waater.
    This bodies have been located in a few locations of Spain as X, Y and Z. The figures X and Y
    show how this bodies of plants look like from satellite.
    
    The source of the images is the pair of satellites Sentinel 2. This satellites haven been operationg since X year providing remote sensing capablities from the range X, Y of the earth.
    This satellite can pass above the same point every X days allowing researchers and companies to
    track specificic locations to study how they change on time.
    This satellite has 12 different bands or wavelengths where it can see. The bands are X, blah blah
    each one of them are useful for different fields.
    Examples of this are for example the tracking of crops as plant can be measured as X.
    Another use is the tracking of water levels of different lakes or the effect of flooding and
    natural catasprohies.
    
    Remote sensing has it's own set of tools, approaches and problems.
    Some research is approached as a traditional computer vision problem were the researcher will
    generated their own heuristics to be able to track the metric they are intrested into.
    An example of this is the NVDI metric which is a simple calculation around some of the bands of 
    sentinel which results in a relly useful tool to identify clorophil based life and how they
    propagate.
    One of the problems of the previous approach is it own simplicity, it's really good at measuring
    plants but can provide not really useful values for other cases that are not related.
    Machine learning techniques can be useful for remote sensing to avoid the problem of false
    positives in simple heuristics.
    
    Satellite imaging provides an enormous quantity of data. To be specific about the sentinel-2
    platform, this generates what is called tiles of the same coordinates. This images have a size
    of 10000x10000 pixels with a resolution of X on some bands and a resolution of Y in other bands.
    The amount of labelled data is at the scale of the whole earth, the problem is curating this
    data to something useful.
    
    This research is trying to improve the detection rate of this bodies of plants in lakes. 
    Currently there are some lakes that are being tracked in a manual way where the researchers 
    would apply several filters to detect them. Obviously, this approach can't scale to more lakes
    without more operators and not even consider analising this bodies at a global scale.


    \section*{Related work}
    Interesting libraries for loading images from Sentinel-2:

    https://github.com/sertit/eoreader
    \newline
    Maybe:

    https://pro.arcgis.com/en/pro-app/latest/arcpy/get-started/what-is-arcpy-.htm
    \newline
    This looks really nice as it also allows to overlay on maps natively:

    https://geopandas.org/en/stable/getting\_started/introduction.html
    \newline
    This project is lit!

    https://github.com/satellite-image-deep-learning/techniques

    https://github.com/acgeospatial/awesome-earthobservation-code
    \newline
    Lots of datasets!

    https://github.com/Seyed-Ali-Ahmadi/Awesome\_Satellite\_Benchmark\_Datasets
    \newline
    Interesting survery here:

    https://github.com/VIROBO-15/Transformer-in-Remote-Sensing
    \newline
    To visualize custom maps:

    https://python-visualization.github.io/folium/quickstart.html

    https://github.com/giswqs/geemap
    \newline
    Too many resources for satellite imaging:

    https://github.com/sacridini/Awesome-Geospatial
    \newline
    Is this really useful?

    https://github.com/microsoft/torchgeo
    \newline
    Hmmm video results....

    https://github.com/acgeospatial/awesome-earthobservation-code#visualisation

    \subsection*{Image segmentation}
    This problem can be solved in different ways. The general idea is give the researchers more data
    of this bodies so they can analyse where, why and when they will appear but also how do they
    behave on time.
    
    Image segmentation arises as an obvious approach to the problem. We have images of the whole earth
    if we can have enough images with ground truths we could train a system to segment this bodies in
    each tile so as the satellite is moving around the earth an automated system could run when these
    tiles are being generated in real-time.
    
    The problem of this approach is th generation of enough labelled data to become useful.
    Image segmentation has been traditionally expensive to label due to the nature of the label, as
    you need a value per pixel.
    This can be reduced with new approaches as SAM. SAM allows segmenting anything based on a prior
    value. This prior value can be a point or a bounding box. This tool results really useful as
    the dataset we posses consists of few images with tenths of boundingboxes or the ROI.
    SAM could be used to find the edges of this bounding box so then the problem can be approached
    as a segmentation.
    
    \subsection*{Image retreival}
    

    \section*{Method}

    Doing self-supervised learning could be useful to fight the data scarcity.
    As the dataset will be small, a self-training step will be useful to generate
    a powerful embedding space to the train a classifer:
    https://github.com/satellite-image-deep-learning/techniques#self-supervised-unsupervised--contrastive-learning
    https://ermongroup.github.io/blog/tile2vec/
    https://github.com/vladan-stojnic/CMC-RSSR
    
    With the embeddings of the patch maybe and MLP per pixel can classify the thingy.
    
    
    the previous idea is cool but SAM exists now so maybe it can be used to segment the algae patch
    using a point inside the ROI we have.
    
    the we can direct the search of more patches based on the surronding water. 
    The hypothesis is that water has X conditions that create the growth of this so if this conditions
    are meet then the should this floaty thing somewhere in this water mass.


    \section*{Experiments}
    
    \section*{Future work}


    \section*{Conclusions}


\end{document}
